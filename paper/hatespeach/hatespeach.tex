\documentclass{llncs}
\usepackage[american]{babel}
\usepackage[T1]{fontenc}
\usepackage{times}
\usepackage{graphicx}
\usepackage{hyperref}

%%%%%%%%%%%%%%%%%%%%%%%%%%%%%%%%%%%%%%%%%%%%%%%%%%%%%%%%%%%%%%%%%%%%%%%%
\begin{document}
	
%{\let\thefootnote\relax\footnotetext{Copyright \textcopyright\ 2020 for this paper by its authors. Use permitted under Creative Commons License Attribution 4.0 International (CC BY 4.0). CLEF 2020, 22-25 September 2020, Thessaloniki, Greece.}}

\title{Deep Modeling of Latent Representations for Twitter Profiles on Hate Speech Spreaders Identification Task.}
%%% Please do not remove the subtitle.
\subtitle{Notebook for PAN at CLEF \the\year}

\author{Roberto Labadie Tamayo\inst{1} \and Daniel Castro Castro \inst{1}\and Reynier Ortega-Bueno \inst{2}}
\institute{Universidad de Oriente, Cuba\\
\email{roberto.labadie@estudiantes.uo.edu.cu}, ~~\email{danielcc@uo.edu.cu}
\and PRHLT Research Center, Universitat Politècnica de València, Valencia Spain\\
\email{rortega@prhlt.upv.es} }

\maketitle

\begin{abstract}
Briefly describe the main ideas of your approach.
\end{abstract}


\section{Introduction}
What is the task about and why is it important? Be sure to mention the language(s) covered and cite the task overview paper. ~1 paragraph

What is the main strategy your system uses? ~1 paragraph

What did you discover by participating in this task? Key quantitative and qualitative results, such as how you ranked relative to other teams and what your system struggles with. ~1 paragraph

Have you released your code? Give a URL
\section{Related Works}

Background
In your own words, summarize important details about the task setup: kind of input and output (give an example if possible); what datasets were used, including language, genre, and size. If there were multiple tracks, say which you participated in.

Here or in other sections, cite related work that will help the reader to understand your contribution and what aspects of it are novel.

\section{Model Description}
System overview
Key algorithms and modeling decisions in your system; resources used beyond the provided training data; challenging aspects of the task and how your system addresses them. This may require multiple pages and several subsections, and should allow the reader to mostly reimplement your system’s algorithms.

Use equations and pseudocode if they help convey your original design decisions, as well as explaining them in English. If you are using a widely popular model/algorithm like logistic regression, an LSTM, or stochastic gradient descent, a citation will suffice—you do not need to spell out all the mathematical details.

Give an example if possible to describe concretely the stages of your algorithm.

If you have multiple systems/configurations, delineate them clearly.

This is likely to be the longest section of your paper.
\subsection{Profile Modeling}
\subsection{Impostor Method}
\section{Experiments and Results}
Experimental setup
How data splits (train/dev/test) are used.

Key details about preprocessing, hyperparameter tuning, etc. that a reader would need to know to replicate your experiments. If space is limited, some of the details can go in an Appendix.

External tools/libraries used, preferably with version number and URL in a footnote.

Summarize the evaluation measures used in the task.

You do not need to devote much—if any—space to discussing the organization of your code or file formats.

Results
Main quantitative findings: How well did your system perform at the task according to official metrics? How does it rank in the competition?

Quantitative analysis: Ablations or other comparisons of different design decisions to better understand what works best. Indicate which data split is used for the analyses (e.g. in table captions). If you modify your system subsequent to the official submission, clearly indicate which results are from the modified system.

Error analysis: Look at some of your system predictions to get a feel for the kinds of mistakes it makes. If appropriate to the task, consider including a confusion matrix or other analysis of error subtypes—you may need to manually tag a small sample for this.

\section{Conclusion and Future Works}


Conclusion
A few summary sentences about your system, results, and ideas for future work.



\bibliographystyle{splncs03}
\begin{raggedright}
\bibliography{}
\end{raggedright}

\end{document}


%%%%%%%%%%%%%%%%%%%%%%%%%%%%%%%%%%%%%%%%%%%%%%%%%%%%%%%%%%%%%%%%%%%%%%%%

